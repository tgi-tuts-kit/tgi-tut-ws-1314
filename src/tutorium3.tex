\include{common_start}
\include{tutBlatt_methods}
\tutnr{3}
\section{Altlasten}
\subsection{Induktiver Beweis}
\begin{frame}
	\frametitle{Induktive Beweise führen}
	
	\begin{itemize}
		\item Formalia:
		\begin{enumerate}
			\item Induktionsanfang \emph{(i.d.R. n = 1)}
			\item Induktionsvorraussetzung
			\item Induktionsschluss:
			\begin{itemize}
				\item \emph{$n \rightarrow n+1$ (Brauche ich auch $n-1 \implies$ I.A. anpassen)}
				\item \emph{Manchmal hilfreich: $n+1 \rightarrow n$}
				\item \emph{Alle Fälle berücksichtigen!}
			\end{itemize}		 
		\end{enumerate}
		\item Induktion über Wortlänge ($n = |w|$)
		\item Induktion über Ableitungsschritte ($\alpha \implies^n \beta$)
	\end{itemize}
\end{frame}

\section{Pumping Lemma}
\subsection{Pumping Lemma}
\begin{frame}
	\frametitle{Pumping Lemma}
	
%\begin{exampleblock}{Pumping Lemma}
Sei $L$ eine reguläre Sprache. Dann existiert eine Zahl $p \in \mathbb{N}$, sodass für jedes Wort $w \in L$ mit $\left|w \right| > p$ eine Darstellung $$w = xyz$$ existiert, so dass folgende Eigenschaften erfüllt sind:

\begin{enumerate}
\item $\left|y\right| > 0$ $(y \neq \varepsilon)$
\item $\left|xy\right| \leq p$ 
\item Für alle $i \geq 0$ gilt: $xy^iz \in L$
\end{enumerate}
%\end{exampleblock}

\begin{center}
\includegraphics[width=0.55\textwidth]{images/Q116}
\end{center}

\end{frame}
\subsection{Aufgabe 1}
\begin{frame}
	\frametitle{Aufgabe 1}
	Gegeben sei die Sprache $\mathcal{L} = \{w \in \{a,b\}^* \; | \; w \;
	\mbox{enth"alt gleich viele $a$ wie $b$}\}$.
	\begin{enumerate}
		\item Wie lautet das Pumping Lemma? Was genau muss man zeigen, falls man die
		Kontraposition des\\
		Pumping Lemmas verwenden will?
	\end{enumerate}
\end{frame}
\begin{frame}
	\frametitle{Aufgabe 1}
	Gegeben sei die Sprache $\mathcal{L} = \{w \in \{a,b\}^* \; | \; w \;
	\mbox{enth"alt gleich viele $a$ wie $b$}\}$.
	\begin{enumerate}
		\item Wie lautet das Pumping Lemma? Was genau muss man zeigen, falls man die
		Kontraposition des\\
		Pumping Lemmas verwenden will?
		\item Zeigen Sie mit Hilfe des Pumping Lemmas, dass L nicht regul"ar ist!
		\item Zeigen Sie mit Hilfe des Pumping Lemmas, dass die Sprache $\mathcal{L}' =
		\{a^p\; | \; p \; \mbox{Primzahl}\}$ nicht regul"ar ist.
		\item Betrachten Sie nun die Sprache $\mathcal{L}'' = \{a,aab,aaab\}$! Ist diese
		regul"ar? Falls ja, geben Sie einen endlichen\\
		Automaten an, der diese Sprache akzeptiert! Kann man mit dem Pumping Lemma zeigen,
		dass die\\
		Sprache regul"ar ist?
	\end{enumerate}
\end{frame}

\section{Chomsky-Normalform}
\subsection{Chomsky-Normalform}
\frame{
\frametitle{Chomsky-Normalform}
\begin{exampleblock}{Chomsky-Normalform}
Eine CH-2-Grammatik \textit{G} $= (\Sigma,\mathcal{V,S,R) }$ ist in Chomsky-Normalform,  wenn jede Produktion aus $\mathcal{R}$ eine der folgenden Formen hat:
\begin{itemize}
\item  $A \rightarrow BC$
\item  $A \rightarrow a$
\item  $S \rightarrow \varepsilon$ (wenn $\varepsilon \in L$)
\end{itemize}
Wobei gilt $A,B,C\in\mathcal{V}$ und $a \in \Sigma$.
\end{exampleblock}
}


%\subsection{Umwandlung in Chomsky-Normalform}
\frame{
\frametitle{Umwandlung in Chomsky-Normalform}
\begin{enumerate}
\item Für alle $\textcolor{red}{a} \in \Sigma$ und für alle Produktionen, auf deren rechter Seite $\textcolor{red}{a}$ vorkommt 
(außer für $V  \rightarrow \textcolor{red}{a}$, mit $V \in\mathcal{V}$),
wird jedes Vorkommen von $\textcolor{red}{a}$ durch ein \emph{neues} Nichtterminalsymbol $\textcolor{blue}{A}$ ersetzt
%(und $A$ in die Variablenmenge aufgenommen)
und die Produktion $\textcolor{blue}{A} \rightarrow \textcolor{red}{a}$ wird hinzugefügt.
\end{enumerate}

\begin{exampleblock}{Umwandlungsbeispiel (Schritt 1 von 4)}
\begin{columns}[c]
\begin{column}{0.3\textwidth}
\begin{align*}
\mathcal{S} &\rightarrow XY\\
X &\rightarrow \textcolor{red}{a}X\textcolor{red}{b} \mid Z \mid \varepsilon\\
Y &\rightarrow \textcolor{red}{cc}Y \mid \varepsilon\\
Z &\rightarrow X\\
\end{align*}
\end{column}
%
\
\begin{column}{0.05\textwidth}
$\Rightarrow$
\end{column}
%
\begin{column}{0.3\textwidth}
\begin{align*}
\mathcal{S} &\rightarrow XY\\
X &\rightarrow \textcolor{blue}{A}X\textcolor{blue}{B} \mid Z \mid \varepsilon\\
Y &\rightarrow \textcolor{blue}{CC}Y \mid \varepsilon\\
Z &\rightarrow X\\
\textcolor{blue}{A} &\rightarrow \textcolor{red}{a}\\
\textcolor{blue}{B} &\rightarrow \textcolor{red}{b}\\
\textcolor{blue}{C} &\rightarrow \textcolor{red}{c}\\
\end{align*}
\end{column}
\end{columns}
\end{exampleblock}
}

\frame{
\frametitle{Umwandlung in Chomsky-Normalform}
\begin{enumerate}
\setcounter{enumi}{1}
\item
Für Produktionen mit mehr als zwei
Variablen rechts werden \textcolor{blue}{ \emph{neue} Nichterminale} eingeführt
und dazu \textcolor{blue}{passende Produktionen} hinzugefügt.
\end{enumerate}

\begin{exampleblock}{Umwandlungsbeispiel (Schritt 2 von 4)}
\begin{columns}[c]
\begin{column}{0.3\textwidth}
\begin{align*}
\mathcal{S} &\rightarrow XY\\
X &\rightarrow \textcolor{red}{AXB} \mid Z \mid \varepsilon \\
Y &\rightarrow \textcolor{red}{CCY} \mid \varepsilon \\
Z &\rightarrow X\\
A &\rightarrow a\\
B &\rightarrow b\\
C &\rightarrow c\\
\end{align*}
\end{column}
%
\
\begin{column}{0.05\textwidth}
$\Rightarrow$
\end{column}
%
\begin{column}{0.3\textwidth}
\begin{align*}
\mathcal{S} &\rightarrow XY\\
X &\rightarrow \textcolor{blue}{FB} \mid Z \mid \varepsilon \\
Y &\rightarrow \textcolor{blue}{GY} \mid \varepsilon \\
Z &\rightarrow X\\
\textcolor{blue}{F} &\rightarrow \textcolor{blue}{AX}\\
\textcolor{blue}{G} &\rightarrow \textcolor{blue}{CC}\\
A &\rightarrow a\\
B &\rightarrow b\\
C &\rightarrow c\\
\end{align*}
\end{column}
\end{columns}
\end{exampleblock}
}

\begin{frame}
\frametitle{Umwandlung in Chomsky-Normalform}
\begin{enumerate}
\setcounter{enumi}{2}
\item Entfernen von Produktionen der Form $\textcolor{red}{V} \rightarrow \textcolor{red}{\varepsilon}$
f"ur $V \in \mathcal{V}, v \neq \mathcal{S}$ \\ $\Rightarrow$ "`Vorwegnahme"' dieser Produktionen: Für jede Produktion mit einem der obigen $V$
auf der rechten Seite wird eine \textcolor{blue}{neue Produktion} ohne dieses $V$ hinzugefügt.
\end{enumerate}

\begin{exampleblock}{Umwandlungsbeispiel (Schritt 3 von 4)}
\ducttape{-0.25cm}
\begin{columns}[c]
\begin{column}{0.3\textwidth}
\begin{align*}
\mathcal{S} &\rightarrow XY\\
\textcolor{red}{X} &\rightarrow FB \mid Z \mid \textcolor{red}{\varepsilon}\\
\textcolor{red}{Y} &\rightarrow GY \mid \textcolor{red}{\varepsilon}\\
Z &\rightarrow X\\
F &\rightarrow AX\\
G &\rightarrow CC\\
A &\rightarrow a, B \rightarrow b, C \rightarrow c\\
\end{align*}
\end{column}
%
\
\begin{column}{0.05\textwidth}
$\Rightarrow$
\end{column}
%
\begin{column}{0.3\textwidth}
\begin{align*}
\textcolor{red}{\mathcal{S}} &\rightarrow XY \mid \textcolor{blue}{X} \mid  \textcolor{blue}{Y} \mid \textcolor{blue}{\varepsilon}\\
\textcolor{red}{X} &\rightarrow FB \mid Z \mid \textcolor{blue}{\msout{\varepsilon}}\\
\textcolor{red}{Y} &\rightarrow GY \mid \textcolor{blue}{G}\mid \textcolor{blue}{\msout{\varepsilon}}\\
\textcolor{red}{Z} &\rightarrow X\\
F &\rightarrow AX \mid \textcolor{blue}{A}\\
G &\rightarrow CC\\
A &\rightarrow a, B \rightarrow b, C \rightarrow c\\
\end{align*}
\end{column}
\end{columns}
\end{exampleblock}
\end{frame}



\frame{
\frametitle{Umwandlung in Chomsky-Normalform}
\begin{enumerate}
\setcounter{enumi}{3}
\item
Für Produktionen mit einer Variablen rechts werden Zyklen gesucht, 
für gefundene Zyklen werden alle Vorkommnisse aller Variablen des Zyklus durch einen Repräsentanten ausgetauscht. Danach werden triviale Produktionen entfernt.
\end{enumerate}

\begin{exampleblock}{Umwandlungsbeispiel (Schritt 4a von 4)}
\begin{columns}[c]
\begin{column}{0.3\textwidth}
\begin{align*}
\mathcal{S} &\rightarrow XY \mid X \mid Y \mid \varepsilon \\
\textcolor{red}{X} &\rightarrow FB \mid \textcolor{red}{Z} \\
Y &\rightarrow GY \mid G \\
\textcolor{red}{Z} &\rightarrow \textcolor{red}{X}\\
F &\rightarrow AX \mid A\\
G &\rightarrow CC\\
A &\rightarrow a, B \rightarrow b, C \rightarrow c\\
\end{align*}
\end{column}
%
\
\begin{column}{0.05\textwidth}
$\Rightarrow$
\end{column}
%
\begin{column}{0.3\textwidth}
\begin{align*}
\mathcal{S} &\rightarrow XY \mid X \mid Y \mid \varepsilon \\
\textcolor{blue}{X} &\rightarrow FB \mid \msout{\textcolor{blue}{X}} \\
Y &\rightarrow GY \mid G\\
\msout{\textcolor{blue}{X}} &\rightarrow \msout{\textcolor{blue}{X}}\\
F &\rightarrow AX\\
G &\rightarrow CC\\
A &\rightarrow a, B \rightarrow b, C \rightarrow c\\
\end{align*}
\end{column}
\end{columns}
\end{exampleblock}
}

\frame{
\frametitle{Umwandlung in Chomsky-Normalform}
\begin{enumerate}
\setcounter{enumi}{3}
\item
Alle Regeln, die rechts eine einzelne Variable haben, werden durch "`Vorziehen"' der Regeln eliminiert.

Außerdem wird ein neues Startsymbol eingeführt, falls eine Regel $\mathcal{S} \rightarrow \varepsilon$ existiert.
\end{enumerate}

\begin{exampleblock}{Umwandlungsbeispiel (Schritt 4b von 4)}
\ducttape{-0.4cm}
\begin{columns}[c]
\begin{column}{0.3\textwidth}
\begin{align*}
\mathcal{S} &\rightarrow XY \mid \textcolor{red}{X} \mid  \textcolor{red}{Y} \mid \textcolor{red}{\varepsilon}\\
X &\rightarrow FB\\
Y &\rightarrow GY \mid \textcolor{red}{G}\\
F &\rightarrow AX\\
G &\rightarrow CC\\
A &\rightarrow a\\
B &\rightarrow b\\
C &\rightarrow c\\
\end{align*}
\end{column}
%
\
\begin{column}{0.05\textwidth}
$\Rightarrow$
\end{column}
%
\begin{column}{0.3\textwidth}
\begin{align*}
\textcolor{blue}{\mathcal{S'}} & \rightarrow \textcolor{blue}{\mathcal{S} \mid \varepsilon} \\
\mathcal{S} &\rightarrow XY \mid \textcolor{blue}{FB} \mid \textcolor{blue}{GY} \mid \textcolor{blue}{CC}\\
X &\rightarrow FB \\
Y &\rightarrow GY \mid \textcolor{blue}{CC} \\
F &\rightarrow AX\\
G &\rightarrow CC\\
A &\rightarrow a\\
B &\rightarrow b\\
C &\rightarrow c\\
\end{align*}
\end{column}
\end{columns}
\end{exampleblock}
}


\subsection{Umwandlung nach Skript}
\begin{frame}
	\frametitle{Umwandlung in Chomsky-Normalform (nach Skript)}
	\begin{enumerate}
		\item Schritt: neues Startsymbol
		\\ \emph{$S' \rightarrow S | \varepsilon$}
		\\ 
		\item Schritt: Entfernen der $\varepsilon$-Produktionen
		\\ 
		\item Schritt: Entfernen von Kettenregeln
		\\ \emph{z.B. $A \rightarrow B, B \rightarrow c \implies A \rightarrow c$}
		\\ x
		\item Schritt: Überführen in Chomsky-Normalform
		\begin{enumerate}
			\item Terminale ersetzen
			\item Regeln auf 2 Variablen 'kürzen'
		\end{enumerate}
	\end{enumerate}
\end{frame}

\subsection{Aufgabe 2a}
\begin{frame}
	\frametitle{Aufgabe 2}
	Gegeben sei die folgende Grammatik: $\mathcal{G} = (\mathcal{T},\mathcal{V},S,
	\mathcal{P})$ mit\\
	$\mathcal{T} := \{a,b,c,d\}$, $\mathcal{V} := \{S,A,D,M\}$, $\mathcal{P} := \{
	S \rightarrow AMD \; | \; M, A \rightarrow AA \; | \; a, D \rightarrow DD \; | \; d,
	M \rightarrow bMc \; | \; \varepsilon\}$
	\begin{enumerate}
		\item Geben Sie die erzeugte Sprache an!
		\item Wandeln Sie die gegebene kontextfreie Grammatik $\mathcal{G}$ in eine
		"aquivalente kontextfreie Grammatik $\mathcal{G}'$ in\\
		Chomsky-Normalform um, indem sie jeden Schritt durch eine neue Grammatik beschreiben!
	\end{enumerate}
\end{frame}

\section{CYK-Algorithmus}
\subsection{CYK-Algorithmus}
\frame{
\frametitle{CYK Überblick}
CYK ist ein Algorithmus, um das Wortproblem in CH-2 zu lösen. Um den Algorithmus anzuwenden, muss eine Grammatik in Chomsky-Normalform vorliegen.\\
Grundidee zur Überprüfung eines Wortes der Länge $n$:
\begin{itemize}
\item Wir betrachen $V_{i,j} = $ Menge der Nichtterminale, aus denen das Teilwort der Position $i$ bis $j$ abgeleitet werden kann
\item Die Frage, ob $V_{i,j}$ ableitbar ist, lässt sich entscheiden durch Betrachten aller möglichen $V_{i,k}$ und $V_{k+1,j}$
\item $V_{i,i}$ sind trivial
\item Bottom-up lässt sich dadurch $V_{1,n}$ berechnen
\item Ist $S \in V_{1,n}$, so lässt sich das Wort ableiten
\end{itemize}
}

\frame{
\frametitle{CYK Beispiel}
Gegeben sei die Grammatik \textit{G} $= \mathcal{(T,V,}S\mathcal{,P) }$  mit den folgenden Produktionen aus $\mathcal{P}$:
\begin{align*}
S & \rightarrow AX \mid AB \\
X &\rightarrow SB \mid AB \\
A &\rightarrow a\\
B &\rightarrow b\\
\end{align*}
\begin{enumerate}
	\item Lässt sich der CYK-Algorithmus auf $G$ anwenden?
	\item Ist das Wort $aaabbb$ in der Sprache $\mathcal{L}(\textit{G})$?
\end{enumerate}
}

\section{Schluss}
\subsection{Schluss}

\begin{frame}
\frametitle{Bis zum nächsten Mal!}
\begin{center}
  \includegraphics[width=1.5 \textheight]{images/schrodinger.jpg}
\end{center}
\end{frame}

\include{common_end}
